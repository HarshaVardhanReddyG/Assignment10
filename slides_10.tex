%%%%%%%%%%%%%%%%%%%%%%%%%%%%%%%%%%%%%%%%%%%%%%%%%%%%%%%%%%%%%%%
%
% Welcome to Overleaf --- just edit your LaTeX on the left,
% and we'll compile it for you on the right. If you open the
% 'Share' menu, you can invite other users to edit at the same
% time. See www.overleaf.com/learn for more info. Enjoy!
%
%%%%%%%%%%%%%%%%%%%%%%%%%%%%%%%%%%%%%%%%%%%%%%%%%%%%%%%%%%%%%%%

% Inbuilt themes in beamer
\documentclass{beamer}

%packages:
% \usepackage{tfrupee}
% \usepackage{amsmath}
% \usepackage{amssymb}
% \usepackage{gensymb}
% \usepackage{txfonts}

% \def\inputGnumericTable{}

% \usepackage[latin1]{inputenc}                                 
% \usepackage{color}                                            
% \usepackage{array}                                            
% \usepackage{longtable}                                        
% \usepackage{calc}                                             
% \usepackage{multirow}                                         
% \usepackage{hhline}                                           
% \usepackage{ifthen}
% \usepackage{caption} 
% \captionsetup[table]{skip=3pt}  
% \providecommand{\pr}[1]{\ensuremath{\Pr\left(#1\right)}}
% \providecommand{\cbrak}[1]{\ensuremath{\left\{#1\right\}}}
% %\renewcommand{\thefigure}{\arabic{table}}
% \renewcommand{\thetable}{\arabic{table}}      

\setbeamertemplate{caption}[numbered]{}

\usepackage{enumitem}
\usepackage{tfrupee}
\usepackage{amsmath}
\usepackage{amssymb}
\usepackage{gensymb}
\usepackage{graphicx}
\usepackage{txfonts}

\def\inputGnumericTable{}

\usepackage[latin1]{inputenc}                                 
\usepackage{color}                                  \usepackage{textcomp, gensymb}         
\usepackage{array}                                            
\usepackage{longtable}                                        
\usepackage{calc}                                             
\usepackage{multirow}                                         
\usepackage{hhline}                             
\usepackage{mathtools}
\usepackage{ifthen}
\usepackage{caption} 
\providecommand{\pr}[1]{\ensuremath{\Pr\left(#1\right)}}
\providecommand{\cbrak}[1]{\ensuremath{\left\{#1\right\}}}
\renewcommand{\thefigure}{\arabic{table}}
\renewcommand{\thetable}{\arabic{table}}   
\providecommand{\brak}[1]{\ensuremath{\left(#1\right)}}

% Theme choice:
\usetheme{CambridgeUS}

% Title page details: 
\title{Assignment 10} 
\author[CS21BTECH11017]{G HARSHA VARDHAN REDDY (CS21BTECH11017)}
\date{\today}
\logo{\large{AI1110}}


\begin{document}

% Title page frame
\begin{frame}
    \titlepage 
\end{frame}
\logo{}


% Outline frame
\begin{frame}{Outline}
    \tableofcontents
\end{frame}

%\section{Abstract}
%\begin{frame}{Abstract}
%\begin{block}{} This document contains $3^{rd}$ %problem from the chapter $5$ in the book %\textbf{Papoulis Pillai Probability Random %Variables and Stochastic Processes.}
%\end{block}

%\end{frame}


\section{Problem Statement}
\begin{frame}{Problem Statement}

    \begin{block} {Papoulis Pillai Probability Random Variables and Stochastic Processes\\ 
    Exercise : 5-40}  Let $x$ denote the event "the number of failures that precede the $nth$ success" so that $x + n$
represents the total number of trials needed to generate n successes. In that case, the event
${x=k}$ occurs if and only if the last trial results in a success and among the previous
$(x + n -1)$ trials there are $n -1$ successes (or $x$ failures). This gives an alternate formula
for the Pascal (or negative binomial) distribution as follows:
\begin{align}
    P\cbrak{X=k}=\binom{n+k-1}{k}p^n q^k=\binom{-n}{k}p^n (-q)^{k} ,k=0,1,2,\dots \label{1}
\end{align}
Find $\Gamma_z$ and show that $\eta_x=\frac{nq}{p}$ and $\sigma^2_x = \frac{nq}{p^2}$
    \end{block}
\end{frame}
\section{Pascal distribution}
\begin{frame}{Pascal or Negative binomial distribution}
\begin{block}{Pascal distribution}
Pascal random variable describes the number of trials until the $k^{th}$ success, which is why it is sometimes called the "$k^{th} order $ interarrival time for a Bernoulli process". 
Let $X_k$ be a $k^{th} order $ Pascal random variable. Then its PMF is given by
\begin{align}
    p_{X_k}=\binom{n+k-1}{k} p^n q^k \label{2}
\end{align}
\end{block}
\end{frame}

\section{Solution}
\begin{frame}{Solution}
    As 
    \begin{align}
        \Gamma(z)&=\sum_{k=0}^\infty p_{X_k}z^k
    \end{align}
From \eqref{1} and \eqref{2}
\begin{align}
    \Gamma(z)&=\sum_{k=0}^\infty \binom{-n}{k}p^n (-q)^{k}z^k\\
    &=p^n \sum_{k=0}^\infty \binom{-n}{k} (-qz)^k\\
    \implies \Gamma(z)&=p^n (1-qz)^{-n}
    \label{6}
\end{align}
And We know that mean $(\eta_X) =\Gamma^\prime(1)$
and $\Gamma^{\prime\prime}(1) +\eta_X = \sigma_X ^2 + \eta_X ^2$
\end{frame}
\begin{frame}{}
From \eqref{6},
\begin{align}
    \Gamma^\prime(z) &= p^n(-q)(-n)(1-qz)^{-(n+1)}\\
    \implies \Gamma^\prime(1) &= p^n nq (1-q)^{-(n+1)}\\
    \implies \eta_X&=\frac{nqp^n}{p^{(n+1)}}=\frac{nq}{p}........(\text{since} p+q=1)\\
    %\vspace{pt}
    \Gamma^{\prime\prime}(z)&= p^n nq(n+1)q(1-qz)^{-(n+2)}\\
    \implies \Gamma^{\prime\prime}(1)&= p^n n(n+1)q^2(1-q)^{-(n+2)}\\
    &=\frac{n(n+1)p^nq^2}{p^{n+2}}\\
    \implies  \sigma_X ^2 + \eta_X ^2 &=\frac{n(n+1)q^2}{p^{2}} +\eta_X
\end{align}    
\end{frame}
\begin{frame}{}
    \begin{align}
        \implies  \sigma_X ^2 &=\frac{n(n+1)q^2}{p^{2}} +\frac{nq}{p}- \left(\frac{nq}{p}\right)^2\\
        &=\frac{nq^2+npq}{p^2}=\frac{nq(p+q)}{p^2}\\
        \implies \sigma_X^2 &=\frac{nq}{p^2}
    \end{align}
\end{frame}
 
\end{document}